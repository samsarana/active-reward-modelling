\documentclass[11pt, a4paper, bibliography=totoc]{report}
\usepackage{amsmath,amsthm,amssymb, ltablex, listings, enumerate, cancel, bussproofs, tabto, wasysym, algpseudocode, algorithm, savesym, mathtools, physics, xcolor, setspace, appendix, dirtree, ragged2e}
\usepackage[margin=3.5cm]{geometry}
\usepackage[font=footnotesize,labelfont=bf]{caption}
\usepackage[nottoc]{tocbibind} % adds Bibliography to 
\usepackage[linktocpage]{hyperref}

\hypersetup{
	colorlinks,
	linkcolor={red!50!black},
	citecolor={blue!50!black},
	urlcolor={blue!80!black}
}
\doublespacing
\pagestyle{headings}

\newcommand{\nats}{\mathbb{N}}
\newcommand{\reals}{\mathbb{R}}
\renewcommand{\P}[1]{\mathbb{P}\left( #1 \right) }
%\newcommand{\E}[1]{\mathbb{E} \left[ #1 \right] }
\newcommand{\E}[2]{\mathbb{E}_{#1} \left[ #2 \right] }
\newcommand{\V}[2]{\mathbb{V}ar_{#1} \left[ #2 \right]}
\newcommand{\KLD}[2]{\mathrm{KL} \left[ \left. \left. #1 \right|\right| #2 \right] }
\newcommand{\rp}{\hat{r}}
\newcommand{\expbuff}{\mathrm{E}}
\newcommand{\annbuff}{\mathrm{A}}
\newcommand{\repbuff}{\mathrm{D}}

\newcommand{\w}{\mathbf{w}}
\newcommand{\data}{\mathcal{D}}
\newcommand{\vfe}{\mathcal{F}(\data, \theta)}
\newcommand{\F}{\mathcal{F}}
\newcommand{\bepsilon}{\pmb{\epsilon}}
\newcommand{\bmu}{\pmb{\mu}}
\newcommand{\bsigma}{\pmb{\sigma}}
\newcommand{\btheta}{\pmb{\theta}}
\newcommand{\brho}{\pmb{\rho}}
\newcommand{\normal}[3]{\mathcal{N}(#1 \vert #2, #3)}
\newtheorem{claim}{Claim}
\newtheorem{theorem}{Theorem}
\newtheorem{lemma}{Lemma}
\newtheorem{hypothesis}{Hypothesis}
\newtheorem{corollary}{Corollary}
\newtheorem{proposition}{Proposition}
\newtheorem{definition}{Definition}
\newtheorem{assumption}{Assumption}

\begin{document}
\title{ARMA: Active Reward Modelling for Agent Alignment}
\author{Sam Clarke}
\date{September 2019}
\renewcommand{\bibname}{References}
\maketitle

\begin{abstract} % ~200 words

\end{abstract}

\tableofcontents
% say somewhere around here about collaboration with Zac and Angelos
% and where do I say who my supervisor is?

\part{Background}

\chapter{Introduction}

\section{Relation to Material Studied on the MSc Course}

\chapter{Neural Networks}
Or can I fit this in more naturally as part of the narrative? In any case, it need to come before I mention DQN.

\chapter{Reinforcement Learning} % Explain problem, give context
Reinforcement learning (RL) refers simultaneously to a problem, methods for solving that problem, and the field that studies the problem and its solution methods. The problem of RL is to learn what to do---how to map situations to actions---so as to maximise some numerical reward signal \cite[pp.~1-2]{Sutton2018}.

In this section we first introduce the elements of RL informally. We then formalise the RL Problem as the optimal control of incompletely-known Markov decision processes (finite? do I talk about PO?). We give a taxonomy of different RL solution methods and conclude with a description of one such method, Deep Q-Learning (DQN) that is of particular importance in this dissertation.

\section{Elements of Reinforcement Learning}
This subsection will introduce agent, environment, policy, reward signal, value function and [model] informally, similar to S\&B 1.3. Is this necessary or should I skip straight to the formalism?

\section{Finite Markov Decision Processes}
Finite Markov Decision Processes (finite MDPs) are a way of mathematically formalising the RL problem: they capture the most important aspects of the problem faced by an agent interacting with its environment to achieve a goal. We introduce the elements of this formalism: the agent-environment interface, goals and rewards, returns and episodes. Then...

\subsection{The Agent-Environment Interface} \label{subsection:agent_env_interface}
MDPs consist firstly of the continual interaction between an agent selecting actions, and an environment responding by changing state, and presenting the new state to the agent, along with an associated scalar reward. Recall that the agent seeks to maximise this reward over time through its choice of actions.

More formally, consider a sequence of discrete time steps, $t = 1,2,3, \dots$. At each time step $t$, the agent receives some representation of the environment's \textit{state}, $ S_t \in \mathcal{S} $, and chooses an \textit{action}, $ A_t \in \mathcal{A} $. On the next time step, the agent receives reward $ R_{t+1} \in \mathcal{R} \subset \reals $, and finds itself in a new state, $ S_{t+1} $. These interactions repeat over time, giving rise to a \textit{trajectory}, $ \tau $:
\begin{align*}
S_0, A_0, R_1, S_1, A_1, R_2, S_2, A_2, R_3, \dots
\end{align*}
% TODO insert Figure 3.1-like figure from S&B
% TODO ask best practices for figures
% TODO ask best practices for latex commands

A \textit{finite} MDP is one where the sets of states, actions and rewards are finite. In this case, the random variables $ S_t $ and $ R_t $ have well-defined discrete probability distributions which depend only on the preceding state and action. This allows us to define the \textit{dynamics} of the MDP, a probability mass function $ p : \mathcal{S} \times \mathcal{R} \times \mathcal{S} \times \mathcal{A} \mapsto [0,1] $, as follows. For any particular values $ s' \in \mathcal{S} $ and $ r \in \mathcal{R} $ of the random variables $ S_t $ and $ R_t $, there is a probability of these values occurring at time $ t $, given any values of the previous state $ s \in \mathcal{S} $ and action $ a \in \mathcal{A} $:
\begin{align*}
p(s', r \mid s, a) := \P{S_t = s', R_t = r \mid S_{t-1} = s , A_{t-1} = a }.
\end{align*}

A \textit{Markov} Decision Process is one where all states satisfy the Markov property. A state $ s_t $ of an MDP satisfies this property iff:
\begin{align*}
\P{s_{t+1}, r_{t+1} \mid s_t, a_t, s_{t-1}, a_{t-1}, \dots, s_0, a_0 } = \P{s_{t+1} \mid s_t, a_t}.
\end{align*}
This implies that the immediately preceding state $ s_t $ and action $ a_t $ are sufficient statistics for predicting the next state $ s_{t+1} $ and reward $ r_{t+1} $.

\subsection{Goals and Rewards}
The reader may have noticed that we first introduced MDPs as a formalism for an agent interacting with its environment to achieve a goal, yet have since spoken instead of maximising a reward signal $ R_t \in \reals $ over time. Our implicit assumption is the following hypothesis:
\begin{hypothesis}[Reward Hypothesis]
\textit{All of what we mean by goals and purposes can be well thought of as the maximization of the expected value of the cumulative sum of a received scalar signal (called reward).} \cite[p.~53]{Sutton2018}
\end{hypothesis}
However, this hypothesis gives no information about how to construct such a scalar signal; only that it exists. Indeed, recent work has shown that it is far from trivial to do so; possible failure modes include negative side effects, reward hacking and unsafe exploration \cite{Amodei2016}. This is central to the topic of this dissertation---our aim is to improve the sample efficiency of one particular method of reinforcement learning when the reward signal is unknown.

\subsection{Returns and Episodes} \label{rets_and_episodes}
Having asserted that we can express the objective of reinforcement learning in terms of scalar reward, we now formally define this objective. Consider the following objective:
\begin{definition}[(Future discounted) return]
Let a sequence of rewards between time step $ t + 1 $ and $ T $ (inclusive) be $ R_{t+1}, R_{t+1}, \dots, R_T $. Let $ \gamma \in [0, 1] $ be a discount factor of future rewards. Then we define the (future discounted) return of this sequence of rewards \cite[p.~57]{Sutton2018}:
\begin{equation} \label{G_t}
G_t := \sum_{k=0}^{\infty} \gamma^k R_{t+k+1}.
\end{equation}
\end{definition}

One reason for introducing a discount factor is because we would like this infinite sum to converge. Accordingly, we impose the condition that $ \gamma < 1 $ whenever the reinforcement learning task is \textit{continuous}, that is to say, there may be an infinite number of non-zero terms in the sequence of rewards $ \{R_{t+1}, R_{t+2}, R_{t+3}, \dots \} $.

The other kind of task is called \textit{episodic}. Here, interactions between the agent and environment occur in well-defined subsequences, each of which ends in a special \textit{terminal state}. The environment then resets to a starting state, which may be fixed or sampled from a distribution. To adapt the definition in (\ref{G_t}) to this case, we introduce the convention that zero reward is given after reaching the terminal state. This is because we typically analyse such tasks by considering a single episode---either because we care about that episode in particular, or something that holds across all episodes \cite[p.~57]{Sutton2018}. Observe that summing to infinity in (\ref{G_t}) is then identical to summing over the episode, and that the sum is well-defined regardless of the discount factor $ \gamma $.

\subsection{Policies and Value Functions} \label{policy_value_functions}
\textit{Policy} determines the behaviour of the agent. Formally, a policy $ \pi : \mathcal{S} \times \mathcal{A} \mapsto [0,1] $ defines a probability distribution over actions, given a state. That is to say, $ \pi(a \mid s) $ is the probability of selecting action $ a $ if an agent is following policy $ \pi $ and in state $ s $.

The \textit{state-value function} $ v_\pi : \mathcal{S} \mapsto \reals $ for a policy $ \pi $ gives the expected return of starting in a state and following that policy. More formally,
\begin{definition}[State-value function]
	Let $ \pi $ be a policy and $ s \in \mathcal{S} $ be any state. We write $ \E{\pi}{.}$ to denote the expected value of the random variable $ G_t $ as defined in (\ref{G_t}). Then the state-value function (or simply, value function) for policy $ \pi $ is:
	\begin{equation} \label{v_pi}
	v_\pi(s) := \E{\pi}{G_t \mid S_t = s} = \E{\pi}{\sum_{k=0}^{\infty} \gamma^k R_{t+k+1} \mid S_t = s}.
	\end{equation}
\end{definition}

The \textit{action-value function} $ q_\pi : \mathcal{S} \times \mathcal{A} \mapsto \reals $ for a policy $ \pi $ is defined similarly. It gives the expected return of starting in a state, taking a given action, and following policy $ \pi $ thereafter.
\begin{definition}[Action-value function]
	Let $ \pi $ be a policy, $ s \in \mathcal{S} $ be any state and $ a \in \mathcal{A} $ any action. Then the action-value function (or, Q-function) for policy $ \pi $ is:
	\begin{equation} \label{v_pi}
	q_\pi(s, a) := \E{\pi}{G_t \mid S_t = s, A_t = a} = \E{\pi}{\sum_{k=0}^{\infty} \gamma^k R_{t+k+1} \mid S_t = s, A_t = a}.
	\end{equation}
\end{definition}

\subsection{Optimal Policies and Optimal Value Functions} \label{optimal_policy_value_functions}
%TODO I'm confused about optimal policy. On Spinningup it seems to be defined as the policy that maximises the v_\pi(s_0). But S&B say it's the policy that maximises v_\pi(s) for all s \in S. These two defs are inconsistent bc consider an unreachable state? Spinningup def could say that some policy is optimal despite being bad in unreachable state; S&B def would not...
% http://spinningup.openai.com/en/latest/spinningup/rl_intro.html#the-rl-problem
% p.64 S&B
The problem of Reinforcement Learning is thus to find an optimal policy.

All optimal policies share the same value functions. We call these the \textit{optimal state-value function}, $ v_* $, and the \textit{optimal action-value function} $ q_* $:
\begin{definition}[Optimal state-value function (from {\cite[p.~62]{Sutton2018}})]
	$$ v_*  := \max_\pi v_\pi(s) ~~ \forall s \in \mathcal{S} .$$
\end{definition}
\begin{definition}[Optimal action-value function (from {\cite[p.~63]{Sutton2018}})]
	$$ q_*  := \max_\pi q_\pi(s,a) ~~ \forall s \in \mathcal{S} ~~ \forall a \in \mathcal{A} .$$
\end{definition}

There is a simple connection between optimal Q-function and optimal policy that will be used in Section \ref{RL_solution_methods}:
\begin{claim} \label{Q_claim}
	If an agent has $ q_* $, then acting according to the optimal policy when in some state $ s $ is as simple as finding the action $ a $ that maximises $ q_*(s,a) $ \cite[p.~64]{Sutton2018}.
\end{claim}

\subsection{Bellman Equations}
These value functions obey special recursive relationships called Bellman equations. The equations are proved by formalising the simple idea that the value of being in a state is the expected reward of that state, plus the value of the next state you move to. Each of the four value functions defined in Sections \ref{policy_value_functions} and \ref{optimal_policy_value_functions} satisfy slightly different equations. We prove the Bellman equation for the value function and state the remaining three for completeness.

\begin{proposition}[Bellman equation for $ v_\pi $ {\cite[p.~59]{Sutton2018}}]
	Let $ \pi $ be a policy, $ p $ the dynamics of an MDP, $ \gamma $ a discount factor and $ v_\pi $ a state-value function. Then:
	\begin{align}
		v_\pi(s) = \underset{\substack{a \sim \pi(. \mid s) \\ s', r \sim p(., . \mid s, a) }}{\mathbb{E}} \left[ r + \gamma v_\pi(s') \right]
	\end{align}
\end{proposition}
\begin{proof}
	\begin{align*}
		v_\pi(s) &:= \E{\pi}{G_t \mid S_t = s} \\
		         &= \E{\pi}{\sum_{k=0}^{\infty} \gamma^k R_{t+k+1} \mid S_t = s} \\
		         &= \E{\pi}{R_{t+1} + \gamma G_{t+1}} \\
		         &= \sum_{a \in \mathcal{A}} \pi(a \mid s) \sum_{\substack{s' \in \mathcal{S} \\ r \in \mathcal{R}}} p(s', r \mid s, a) \left[ r + \gamma \E{\pi}{G_{t+1} \mid S_{t+1} = s'} \right] \\
		         &= \sum_{a \in \mathcal{A}} \pi(a \mid s) \sum_{\substack{s' \in \mathcal{S} \\ r \in \mathcal{R}}} p(s', r \mid s, a) \left[ r + \gamma  v_\pi(s') \right] \\
		         &= \underset{\substack{a \sim \pi(. \mid s) \\ s', r \sim p(., . \mid s, a) }}{\mathbb{E}} \left[ r + \gamma v_\pi(s') \right]
	\end{align*}
\end{proof}

\begin{proposition}[Bellman equation for $ q_\pi $]
	Let $ \pi $ be a policy, $ p $ the dynamics of an MDP, $ \gamma $ a discount factor and $ q_\pi $ an action-value function. Then:
	\begin{align}
	q_\pi(s, a) = \underset{s', r \sim p(., . \mid s, a)}{\mathbb{E}} \left[ r + \gamma \underset{a' \sim \pi(.\mid s')}{\mathbb{E}}\left[q_\pi(s', a')\right] \right]
	\end{align}
\end{proposition}

\begin{proposition}[Bellman equation for $ v_* $ {\cite[p.~63]{Sutton2018}}]
	Let $ p $ be the dynamics of an MDP, $ \gamma $ a discount factor and $ v_* $ an optimal value function. Then:
	\begin{align}
	v_*(s) = \max_{a \in \mathcal{A}} ~ \underset{s', r \sim p(., . \mid s, a)}{\mathbb{E}} \left[ r + \gamma v_*(s') \right]
	\end{align}
\end{proposition}

\begin{proposition}[Bellman equation for $ q_* $ {\cite[p.~63]{Sutton2018}}] \label{bellman_q*}
	Let $ p $ be the dynamics of an MDP, $ \gamma $ a discount factor and $ q_* $ an optimal Q-function. Then:
	\begin{align}
	q_*(s, a) = \underset{s', r \sim p(., . \mid s, a)}{\mathbb{E}} \left[ r + \gamma ~ \underset{a' \in \mathcal{A}}{\max}\left[q_*(s', a')\right] \right]
	\end{align}
\end{proposition}

\section{Reinforcement Learning Solution Methods} \label{RL_solution_methods}
One method of solving the reinforcement learning problem is to explicitly solve a set of Bellman optimality equations. For example, in a finite MDP with $ n $ states and $ m $ actions, the Bellman equations for $ q_* $ are a set of $ n\cdot m $ equations in $ n\cdot m $ unknowns\footnote{This assumes that the agent can take any action in any state.}. Given the dynamics $ p $ of the MDP, standard techniques for solving systems of equations can be applied. Then, via Claim (\ref{Q_claim}), the agent has an optimal policy \cite[p.~64]{Sutton2018}.

However, in reality, we rarely have access to $ p $, or sufficient computational resources to solve this system of equations exactly \cite[p.~66]{Sutton2018}. Thus, the literature on RL solution methods focuses on finding approximate solutions.

In this section, we give a brief and incomplete taxonomy of RL solution methods, most of which give approximate solutions. Then, we focus on one such method, the deep Q-network, which is important for the rest of this thesis.

\subsection{Taxonomy of RL Solution Methods} \label{subsection:rl_taxonomy}
The following taxonomy draws on those given in \cite{Sutton2018} and \cite{Achiam2019}.

\dirtree{%
	.1 RL Solution Methods.
	.2 Tabular Solution Methods.
	.3 Dynamic Programming.
	.4 Policy Iteration.
	.4 Value Iteration.
	.3 Monte Carlo Methods.
	.4 MC Prediction.
	.4 MC Control.
	.3 Temporal-Difference Learning.
	.4 Sarsa.
	.4 Q-Learning.
	.2 Approximate Solution Methods.
	.3 Deep RL.
	.4 Model-Free.
	.5 Policy Optimisation.
	.6 Policy Gradient.
	.6 A2C / A3C.
	.6 PPO.
	.6 TPRO.
	.5 Q-Learning.
	.6 \textbf{DQN}.
	.6 C51.
	.5 Policy Optimisation + Q-Learning.
	.6 DDPG.
	.6 TD3.
	.6 SAC.
	.4 Model-Based.	
	.5 Given the Model.
	.6 AlphaZero.
	.5 Learn the Model.
	.6 I2A.
}
% design choices; trade-offs?
% model-based v model-free

\subsection{Deep Q-network} \label{DQN}
The deep Q-network (DQN) agent approximates $ q_* $ using a deep convolutional neural network $ Q(s,a; \theta) $ as the function approximator \cite{Mnih2015}. $ \theta $ are the parameters (weights) of the neural network, which is called a Q-network. It can be trained by performing gradient descent on the parameters $ \theta_i $ at iteration $ i $ to reduce the mean-squared error between $ Q(s,a; \theta_i) $ and the Bellman equation for $ q_*(s,a) $, given in Proposition (\ref{bellman_q*}). Since we do not have access to the true value of this target, we instead use approximate target values $ y = r + max_{a'} ~ Q(s', a' ; \theta_i^-) $, with $ \theta_i^- $ some previous network parameters.

One could then perform standard gradient descent on this loss function using the experience collected by the agent $\langle (s_t, a_t, r_{t+1}, s_{t+1}) \rangle_{t=0}^T $ as data, as one would do in the supervised learning setting. However, there are three important differences in the reinforcement learning setting: correlations both (i) in the data set, and (ii) between $ Q(s,a ; \theta_i) $ and the targets. Furthermore, (iii) updates to $ Q $ may change the policy and thus change the data distribution. This leads to instability in training. To address this, the authors propose two algorithmic innovations. Firstly, instead of training on experience in the order that it is collected, the agent maintains a buffer of experience $ D_t = \{ e_1, e_2, \dots, e_t \} $ where $ e_t = (s_t, a_t, r_{t+1}, s_{t+1}) $. When making learning updates, drawing minibatches uniformly at random from this buffer breaks correlations in the experience sequence and smooths over changes in the data distribution, alleviating problems (i) and (iii). This is termed \textit{experience replay}. Secondly, to reduce correlations between $ Q $ and the targets and alleviate problem (ii), the approximate target values are updated to match the parameters $ Q $ only every $ C $ steps for some hyperparameter $ C > 1 $.

With these changes, we arrive at a loss function $ \ell_i(\theta_i) $ for each learning update $ i $:
\begin{align*}
\ell_i(\theta_i) &= \E{ s,a,r }{ (\E{s'}{y \mid s,a} - Q(s,a;\theta_i))^2 } \\
                 &= \E{ s,a,r }{ \E{s'}{y - Q(s,a;\theta_i) \mid s,a } ^2 } \\
                 &= \E{ s,a,r }{ \E{s'}{(y - Q(s,a;\theta_i))^2 \mid s,a}  - \V{s'}{y - Q(s,a;\theta_i)) \mid s,a}   } \\
                 &= \E{s,a,r,s'}{(y - Q(s,a;\theta_i))^2} - \E{s,a,r}{\V{s'}{y}}
\end{align*}
where the expectations and variances are with respect to samples from the experience replay. This loss function is then optimized by stochastic gradient descent with respect to the network parameters $ \theta_i $. Note that the final term is independent of these parameters, so we can ignore it. Finally, the authors found that stability is improved by clipping the error term $ y - Q(s,a;\theta_i) $ to be between $ -1 $ and $ 1 $.

We summarise this training procedure in Algorithm \ref{alg:dqn}. To ensure adequate exploration, the agent's policy is $ \epsilon $-greedy with respect to the current estimate of the optimal action-value function.

\begin{algorithm}
	\caption{Deep Q-learning with experience replay.}
	\label{alg:dqn}
	\begin{algorithmic}[1]
		\State Initialise replay memory $ D $ to capacity $ N $
		\State Initialise neural network $ Q $ with random weights $ \theta $ as approximate optimal action-value function
		\State Initialise neural network $ \hat{Q} $ with identical weights $ \theta^- = \theta $ as approximate target action-value function
		\State Reset environment to starting state $ s_0 $
		\For{$ t=0, \dots, T $}
		\State With probability $ \epsilon $ execute random action $ a_t $
		\State otherwise execute action $ a_t = \arg\max_a Q(s_t, a ; \theta) $
		\State Observe next state and corresponding reward $ s_{t+1}, r_{t+1} \sim p(.,.\mid s_t,a_t) $
		\State Store transition $ (s_t, a_t, r_{t+1}, s_{t+1}) $ in $ D_t $
		\State Randomly sample minibatch of transitions $ (s_j, a_j, r_{j+1}, s_{j+1}) \sim D_t $
		\State Set $ y_j = \begin{cases}
		                   r_{j+1} \text{ if episode terminates at step $ j+1 $} \\
		                   r_{j+1} + \gamma \max_{a'} \hat{Q}(s_{j+1}, a_j ; \theta^- ) \text{ otherwise}
		\end{cases}$
		\State Do gradient descent on $ (y_j - Q(s_j, a_j ; \theta))^2 $ w.r.t network parameters $ \theta $
		\State Every $ C $ steps set $ \theta^- = \theta $
		\EndFor
	\end{algorithmic}
\end{algorithm}

Note that line 11 assumes we are training DQN to perform an episodic task, hence the first case which follows the convention given in Section \ref{rets_and_episodes} whereby zero reward is given for all states after the terminal state. If the task were instead continuing, line 11 would simply be $ y_j = r_{j+1} + \gamma \max_{a'} \hat{Q}(s_{j+1}, a_j ; \theta^- ) $.

%TODO do I need to describe neural networks too?

\section{Reinforcement Learning from Unknown Reward Functions}
So far, we have assumed that as the agent interacts with the environment, it receives both information about the next state and the associated scalar reward. This presents an obvious challenge if we want to apply RL to solve real-world problems: since the world does not give scalar rewards, it seems we would have to manually specify a reward function mapping states of the world to rewards. For complex or poorly defined goals, this is difficult to do. If we try instead to design an approximate reward function, an agent optimizing hard for this objective will do so at the expense of satisfying our preferences \cite[p.~1]{Christiano2017}.

To circumvent this issue, a growing body of work studies how to do RL from various forms of human \textit{feedback}, instead of an explicit reward function. Three main feedback methods have been studied, all of which involve a human-in-the-loop providing information to the agent about the desired behaviour. Firstly, an agent may learn from expert demonstrations. This may involve using demonstrations to infer a reward function, an approach known as \textit{inverse reinforcement learning} \cite{Ng2000, Ziebart2008}. One can then use a standard RL algorithm on this recovered reward function. Other methods involve training a policy directly from demonstrations, referred to as \textit{imitation learning} \cite{Ho2016, Hester2017}.

Secondly, an agent may learn from feedback on its current policy in the form of scalar rewards \cite{Knox2009, Warnell2017}. Instead of providing a set of demonstrations, the human-in-the-loop observes the agent's behaviour and gives an appropriate reward. If it is assumed that the human provides this reinforcement according to some latent reward function $ \hat{r} : \mathcal{S} \times \mathcal{A} \mapsto \reals $, then standard supervised learning techniques can be applied to model this function. The agent can then select actions so as to maximise modelled reward.

Finally, an agent may learn from binary preferences over trajectories \cite{Wilson2012, Christiano2017}. As with the \textit{policy feedback} method, the human-in-the-loop observes the agent's behaviour. However, instead of giving scalar rewards, she is periodically presented with a pair of trajectories and must indicate which they prefer\footnote{The human also has the option of expressing indifference, or that the trajectories are incomparable.}. In a similar fashion to the above, we can model the human's preferences $ \hat{r} : \mathcal{S} \times \mathcal{A} \mapsto \reals $. Then, standard RL algorithms can be applied to maximise the expected (modelled) reward over time.
%TODO is this last paragraph too specfic since it only applies to the reward learning from trajectory preferences in the Deep RL case?

Each of these methods have shown promising results, and some of the advantages and disadvantages are summarised in Table \ref{table:1}.

\small\singlespacing\RaggedLeft
\begin{tabularx}{\textwidth}{ 
		|X|X|X|X|
	}
	\hline 
\textbf{Property} & \textbf{Trajectory preferences} & \textbf{Expert demonstrations} & \textbf{Policy feedback} \\ 
\hline 
Demandingness for human & Human only needs to judge outcomes & Human needs to perform task (expertly) & Human needs to provide suitable scalar rewards \\ 
\hline 
Upper bound on performance? & Superhuman performance is possible & Impossible to significantly surpass performance of expert & Superhuman performance is possible \\ 
\hline 
Suitability to exploration-heavy tasks & Limited\footnote{\label{footnote:exploration}The human can only give feedback on states visited by the agent. If the agent does not explore well, this limits the amount of information the human can convey. Since exploration is determined by the inferred reward func } & Well suited, since demonstrations can guide exploration & Limited\footnote{See footnote \ref{footnote:exploration}} \\ 
\hline 
Communication efficiency & On the order of hundreds of bits per human hour & Much richer in information than trajectory preferences \footnote{\cite{Ibarz2018} show that demonstrations half the amount of human time required to achieve the same level of performance} & Scalar rewards provide richer information than binary preferences over trajectories, but not as rich as demonstrations \\ 
\hline 
Computational efficiency (in simple Atari environments) & On the order of 10 million RL time steps & On the order of 10 million RL time steps & On the order of thousands of learning time steps \footnote{Note that the work which prototypes this method trains, with an unspecified amount of compute, a deep autoencoder to extract 100 features from the Atari game screen \textit{before} commencing the RL stage. The other methods do not do such pretraining, thus the reported results do not allow for a fair comparison.} \\ 
\hline
\caption{Summary of the properties of using different forms of human feedback in RL without a reward function.} \label{table:1}
\end{tabularx}
\normalsize\doublespacing\justify

Since this dissertation concerns reward learning from trajectory preferences specifically, the remainder of this section explains in more detail the algorithms used to do so. There are two branches of work on this question: that which learns a reward function based on handcrafted features of the environment state, and that which takes the deep learning approach of training a reward model end-to-end without handcrafted features.

\subsection{Reward Learning from Trajectory Preferences with Handcrafted Feature Transformations}
% Summarise the contributions of the 2017,18,19 papers by Sadigh's lab.
% (+Batch) Active Preference-Based Learning of Reward Functions (ignore the active part here)
% DemPref Learning Reward Functions by Integrating Human Demonstrations and Preferences

\subsubsection{Active Preference-Based Learning of Reward Functions}

\subsubsection{Batch Active Preference-Based Learning of Reward Functions}

\subsubsection{DemPref}

\subsection{Reward Learning from Trajectory Preferences in Deep RL}

\subsubsection{Setting}
The agent-environment interface is as described in Section \ref{subsection:agent_env_interface}, with the modification that on step $ t $, instead of receiving reward $ R_{t+1} \in \mathcal{R} \subset \reals $, there is an \textit{annotator} who expresses preferences between \textit{trajectory segments}, or \textit{clips}. A clip is a finite sequence of states and actions $ ((s_0,a_0), (s_1,a_1),\dots,(s_{k-1},a_{k-1})) \in (\mathcal{S} \times \mathcal{A})^k $. If the annotator prefers some clip $ \sigma^1 $ to another clip $ \sigma^2 $, write $ \sigma^1 \succ \sigma^2 $. The annotator may also be indifferent between the clips, in which case we write $ \sigma^1 \sim \sigma^2 $ The agent does not see the annotator's implicit reward function $ r : \mathcal{S} \times \mathcal{A} \mapsto \mathcal{R} $, and must instead use the preferences expressed by the annotator to maximise $ r $ over time. This is the goal of reward learning from trajectory preferences.
%TODO need to define \succ in terms of r(.,.) ?
If the annotator could write down their true $ r $, then clearly we could perform traditional RL instead of taking the reward learning approach. However, as we noted above, we are interested in applying RL to tasks for which we do not have the true $ r $, which is when reward learning is useful. Nonetheless, for the purposes of quantitatively evaluating the method, we consider tasks for which we do have access to the true $ r $.

\subsubsection{Method}
The method has two components: a policy $ \pi : \mathcal{S} \times \mathcal{A} \mapsto [0,1] $ and an estimate of the annotator's reward function, $ \rp : \mathcal{S} \times \mathcal{A} \mapsto \mathcal{R} $, called a \textit{reward model} or \textit{reward predictor}. Both components are parametrised by a deep neural network. The agent learns to maximise the annotator's implicit reward function over time by iterating through the following three processes:
\begin{enumerate}
	\item Reinforcement learning by a traditional deep RL algorithm whereby policy $ \pi $ interacts with the environment for $ T $ steps and updates its parameters to optimise $ \rp $ over time. Agent experience $ \mathrm{E} = ((s_0, a_0), (s_1, a_1),\dots, (s_{T-1}, a_{T-1}) ) $ is stored.
	\item Select pairs of clips $ (\sigma^1, \sigma^2) $ from $ \expbuff $, request a preference $ \mu $ from the annotator on each sampled pair, and add the labelled pair to the annotation buffer $ \annbuff $.
	\item Supervised learning to train the reward model on $ \annbuff $, the preferences expressed by the annotator so far.
\end{enumerate}
More detail on each process is provided below.

\subsubsection{Process 1: Training the policy} %TODO make sure I mention q-learning v policy optimisation distinction in section \ref{subsection:rl_taxonomy}
As mentioned, process 1 is akin to traditional RL except $ \rp $ is used in place of the environment reward $ r $. There are two subtleties to mention. Firstly, since $ \rp $ is learned while RL is taking place, \cite{Christiano2017} prefer policy optimization methods over Q-learning, as these have been successfully applied to RL tasks with a non-stationary reward function \cite{Ho2016}. Specifically, they use A2C \cite{Mnih2016} and TPRO \cite{Schulman2015}. However, the follow up paper \cite{Ibarz2018} uses a Q-learning method\footnote{The reason for deviating from the recommendation in \cite{Christiano2017} is that \cite{Ibarz2018} combine reward learning from trajectory preferences and expert demonstrations, and DQfD is state-of-the-art for the latter problem.}, DQfD \cite{Hester2017}, and it is not clear that performance is impaired. Hence, the literature is ambiguous on whether a non-stationary reward function is necessarily problematic for Q-learning. %TODO link to the section of my method where I say that I used DQN; motivate why I did this; say that reinitialising agent should make it okay.

Secondly, since the reward model $ \rp $ is trained only on pairwise comparisons, its scale is underdetermined. Previous work therefore proposes periodically normalising $ \rp $ to have zero mean and constant standard deviation over the examples in $ \annbuff $. This is crucial for training stability since deep RL is sensitive to the scale of rewards. %TODO perhaps comment here or later on what Zac said that scaling rewards can actually in some cases change optimal policy...
%TODO comment here on later on the subtelties of normalisation when using an ensemble for your rewards.

\subsubsection{Process 2: Selecting and annotating clip pairs}
Previous work uses two methods for selecting clip pairs. \cite{Ibarz2018} sample uniformly at random from $ \expbuff $. \cite{Christiano2017} train an ensemble of three reward predictors and select the clip pairs with the maximum standard deviation across the ensemble. Roughly this favours clip pairs on which the model is most uncertain, with the hope of improving sample efficiency (that is to say, being able to learn $ \rp $ with fewer labels from the annotator). However, they found that relative to random selection, this sometimes impaired performance and slowed down training. In Section ** we consider what caused this.

The selected clip pairs $ \sigma^1, \sigma^2 $ are then annotated with a label $ \mu $, indicating which clip is preferred. $ \mu $ is a distribution over $ \{1,2\} $ where $ \mu(1) = 1, \mu(2) = 0 $ if $ \sigma^1 \succ \sigma^2 $; $ \mu(1) = 0.5, \mu(2) = 0.5 $ if $ \sigma^1 \sim \sigma^2 $; or $ \mu(1) = 0, \mu(2) = 1 $ if $ \sigma^2 \succ \sigma^1 $. The triples $ (\sigma^1, \sigma^2, \mu) $ are then added to $ \annbuff $. The majority of previous work uses a \textit{synthetic annotator} rather than an actual human. Labels are simply generated according to the (hidden) ground truth reward function $ r $, where $ \sigma^1 \succ \sigma^2 $ if $ \sum_t r(s_t^1, a_t^1) > \sum_t r(s_t^1, a_t^1) $; $ \sigma^1 \sim \sigma^2 $ if $ \sum_t r(s_t^1, a_t^1) = \sum_t r(s_t^1, a_t^1) $; and $ \sigma^2 \succ \sigma^1 $ otherwise. This facilitates quicker experimentation and more clear performance metrics (by using hidden ground truth reward function to evaluate agent performance and reward model alignment).

\subsubsection{Process 3: Training the reward model}
The training of $ \rp $ is based on the following assumption:
\begin{assumption} \label{assumption:1}
	The annotator's probability of preferring clip 1 to clip 2, $ \hat{P}(\sigma^1 \succ \sigma^2) $, depends exponentially on the value of $ \hat{r} $ summed over the clips.
\end{assumption}
This allows us to write:
\begin{equation} \label{eq:1}
\hat{P}(\sigma^1 \succ \sigma^2) = \frac{\exp \sum_t \hat{r}(s_t^1, a_t^1)}{\exp \sum_t \hat{r}(s_t^1, a_t^1) + \exp \sum_t \hat{r}(s_t^2, a_t^2)}
\end{equation}
We can then fit the parameters of $ \rp $ by treating the problem as binary classification. In other words, we can use standard supervised learning techniques to optimize the parameters of $ \rp $ so as to minimise the cross-entropy loss between the predictions in \ref{eq:1} and the annotator's labels.
\begin{equation}
\text{loss}(\hat{r}) = -\sum_{(\sigma^1, \sigma^2, \mu) \in \annbuff} \mu(1) \log \hat{P}(\sigma^1 \succ \sigma^2) + \mu(2)\log \hat{P}(\sigma^2 \succ \sigma^1)
\end{equation}
Assumption \ref{assumption:1} follows the Elo rating system developed for chess \cite{elo1978rating}. Given a zero-sum game and two players with a scalar rating, Elo specifies a mapping from player ratings (in our case: reward) to the probability of each player winning (in our case: the probability of each clip being preferred by the annotator).

\chapter{Uncertainty in Deep Learning}
% Talk about Yarin's thesis, BNNs; maybe Ian Obsband paper for good measure (esp. if I end up using ensemble instead of MC-Dropout?)
Standard deep learning models output point estimates. For example, a model trained to classify pictures of dogs according to their breed takes a picture of a dog and outputs its predicted breed. However, what will the model do if it is given a picture of a cat? \cite{Gal2017a}.

We probably want the model to be able to recognise that this is an out of distribution example, and request more training data, or simply say that it doesn't know the answer. However, since standard deep learning models output only point estimates, the model will just go ahead and classify the cat as some breed of dog, just as confidently as any other input.

Thus, the field of Bayesian Deep Learning aims to equip neural networks with the ability to output a point estimate along with its uncertainty in that estimate. Historically, many of the attempts to do so were not very practical. For example, one algorithm, Bayes by Backprop, requires doubling the number of model parameters, making training more computationally expensive, and is very sensitive to hyperparameter tuning. However, recent techniques allow almost any network trained with a stochastic regularisation technique, such as dropout, to, given an input, obtain a predictive mean and variance (uncertainty), without any complicated augmentation to the network \cite[p.~15]{Gal2017a}.

\section{Bayesian Neural Networks}
From Andreas:
2.3 Bayesian Neural Networks (BNN)
In this paper we focus on BNNs as our Bayesian model because they scale well to high dimensional
inputs, such as images. Compared to regular neural networks, BNNs maintain a distribution over
their weights instead of point estimates. Performing exact inference in BNNs is intractable for any
reasonably sized model, so we resort to using a variational approximation. Similar to Gal et al. [10],
we use MC dropout [9], which is easy to implement, scales well to large models and datasets, and is
straightforward to optimise.

\section{Model Uncertainty in BNNs}

\chapter{Active Learning}
% BALD (Houlsby: Bayesian Active Learning for Classification and Preference Learning)
% Adapting it to the deep case (Yarin's thesis/Image Data paper)
\section{Acquisition Functions}

\subsection{Max Entropy}

\subsection{Variation Ratios}

\subsection{Mean STD}

\subsection{BALD}

\section{Applying Active Learning to RL without a reward function}
Discussion of previous work.

\subsection{APRIL}
% APRIL

\subsection{Active Preference-Based Learning of Reward Functions with handcrafted feature transformations}
% Summarise the *active learning parts* of the 2017,18,19 papers by Sadigh's lab (I've already summarised the 
% reward learning parts, above)
% it probably doesn't make sense to have a subsubsection for each, as I assume the active parts are basically the same?
% but work this out once you've worked out what the differences are

\subsubsection{Active Preference-Based Learning of Reward Functions}

\subsubsection{Batch Active Preference-Based Learning of Reward Functions}

\subsubsection{DemPref}
% Learning Reward Functions by Integrating Human Demonstrations and Preferences (DemPref) (they use the `Volume Removal Method' to do Active Learning)

\subsection{Deep RL from Human Preferences}
% Christiano: summarise what didn't work, and what might have gone wrong (how to frame this depends on the results that I eventually get to)


\part{Innovation}

\chapter{Method}
% Here I'll describe the training protocol I used (mostly Ibarz but without the demos)
% I can leave gorey detail to Experiments/Appendix
% I should have already summarised the high level approach in Background
% And I'll describe the different methods of Active Learning/uncert estimates that I tried

\section{Possible failure modes of active reward modelling}
Before developing our method, it was important to consider the possible failure modes of the previous attempt to apply active learning to reward modelling. \cite{Christiano2017} compute the standard deviation of each clip pair $(\sigma^1, \sigma^2)$ across the ensemble. In the language of acquisition functions, they use $ a_{mean\_std}(\sigma^1, \sigma^2) $, which in this setting is written as:
\begin{align*}
a_{mean\_std}(\sigma^1, \sigma^2) = \sqrt{\frac{1}{3}\sum_{i=1}^{3}\left(\hat{P_i}(\sigma^1 \succ \sigma^2) - \overline{\hat{P}(\sigma^1 \succ \sigma^2)}\right)^2},
\end{align*}
where $ \hat{P_i}(\sigma^1 \succ \sigma^2) $ is the prediction $ \hat{P}(\sigma^1 \succ \sigma^2) $ according to component $ i $ of the ensemble, and $ \overline{\hat{P}(\sigma^1 \succ \sigma^2)} $ is the average of these predictions.

The quality of uncertainty estimates and active learning method is one possible failure mode. Indeed, \cite[p.~6]{Christiano2017} explain the failure of their implementation as due to $ a_{mean\_std}(\sigma^1, \sigma^2) $ ``crude approximation'' to reward predictor uncertainty.

Additionally, there are three features of the application of active learning to reward modelling that are not present in the standard supervised learning case. Firstly, the pool dataset is not known in advance, but is gathered as the RL agent encounters new states. This seems to present an exploration problem: naively, the agent's exploration is guided by the current reward function estimate, $ \rp $. Thus, the agent may not explore novel states and collect diverse clip pairs for the pool dataset. If the pool dataset is not diverse, it may be hard to beat the random acquisition baseline with active learning, since the data may all be more or less equally informative.
%TODO do I need to define diverse more rigorously

Secondly, since the objects that we acquire are \textit{clip pairs} rather than, for example, single images, yet the (growing) dataset is comprised of single clips, evaluating the acquisition function over the dataset is a complexity $ O(n^2) $ operation, for $ n $ the number of clips acquired. Therefore, the standard active learning procedure of evaluating the acquisition function on each point in the pool dataset and picking that which maximises it, quickly becomes unfeasible as the dataset grows in size. To get around this issue, in order to acquire $ k $ clip pairs, \cite{Christiano2017} randomly sample $ 10k $ clip pairs and select the $ k $ with the highest score according to $ a_{mean\_std} $. However, it is not clear that this method suffices to find clip pairs that are more informative than random acquisition: a factor of 10 may simply be too small.

Thirdly, it is not clear that the standard acquisition functions can be applied out of the box to learning \textit{in the preference space}. Consider the following example: we are deciding whether to acquire clip pair $ c_1 = (\sigma^1, \sigma^2) $ or $ c_2 = (\sigma^3, \sigma^4) $ to acquire. Suppose further that the reward model is uncertain whether $ c_1 $ has label $ 0 $ or $ 0.5 $\footnote{TODO I may need to spell this out more, in terms of draws from the posterior giving something like 0, 0.5, 0, 0.5 ... Can I borrow Yarin's presentation in thesis of a similar case?}; and uncertain whether $ p_2 $ has label $ 0 $ or $ 1 $. Now, $ a_{mean\_std}(c_1) =  $, whereas $ a_{mean\_std}(c_2) =  $, and thus we will acquire $ c_2 $. Yet, when learning in the preference space, using disagreement between models in an ensemble as the basis for an acquisition function may not capture all that we care about. It may be important, for example, to acquire clip pairs that allow the model to make deductions based on transitivity of the preference relation. For suppose that we have already acquired some clip pair $ (\sigma^0, \sigma^1) $. Then acquiring $ (\sigma^1, \sigma^2) $ would in effect give for free the label of $ (\sigma^0, \sigma^2) $, whereas the acquisition of $ (\sigma^3, \sigma^4) $ would not. Thus, we may need a better proxy than disagreement for active learning in the preference space.
%TODO I'm highly uncertain about whether this point is correct

There are other possible failure modes that apply to active learning in general. When performing \textit{batch acquisition}, that is, acquiring the top $ b $ points that maximise an acquisition function \cite{Gal2017b}, may lead to the acquisition of points that are informative individually, but jointly are much less informative than the sum of their parts. In particular, the acquisitions may not be very diverse. Indeed, \cite[p.~8]{Kirsch2019} show that with acquisition size 5, BALD underperforms random acquisition on the EMNIST image dataset \cite{cohen2017emnist} when acquiring new images with acquisition size 5. Specifically, they observe that BALD several classes are under-represented in the acquisitions made by BALD. \cite{Christiano2017} use acquisition sizes of up to 500.

Anything else?

\section{Applying acquisition functions to reward modelling}


\section{ARMA}
In this section we present the training protocol we developed.
\begin{algorithm}
	\caption{ARMA: Active Reward Modelling for Agent Alignment.}
	\label{alg:arma}
	\begin{algorithmic}[1]
		\State Initialise replay memory $ D $ to capacity $ N $
		\State Initialise neural network $ Q $ with random weights $ \theta $ as approximate optimal action-value function
		\State Initialise neural network $ \hat{Q} $ with identical weights $ \theta^- = \theta $ as approximate target action-value function
		\State Reset environment to starting state $ s_0 $
		\For{$ t=0, \dots, T $}
		\State With probability $ \epsilon $ execute random action $ a_t $
		\State otherwise execute action $ a_t = \arg\max_a Q(s_t, a ; \theta) $
		\State Observe next state and corresponding reward $ s_{t+1}, r_{t+1} \sim p(.,.\mid s_t,a_t) $
		\State Store transition $ (s_t, a_t, r_{t+1}, s_{t+1}) $ in $ D_t $
		\State Randomly sample minibatch of transitions $ (s_j, a_j, r_{j+1}, s_{j+1}) \sim D_t $
		\State Set $ y_j = \begin{cases}
		r_{j+1} \text{ if episode terminates at step $ j+1 $} \\
		r_{j+1} + \gamma \max_{a'} \hat{Q}(s_{j+1}, a_j ; \theta^- ) \text{ otherwise}
		\end{cases}$
		\State Do gradient descent on $ (y_j - Q(s_j, a_j ; \theta))^2 $ w.r.t network parameters $ \theta $
		\State Every $ C $ steps set $ \theta^- = \theta $
		\EndFor
	\end{algorithmic}
\end{algorithm}

\section{Acquisition Functions}

\section{Uncertainty Estimates}

\section{Implementation Details}
The training protocol is implemented mostly in Python \cite{van1995python}. We use gradient descent to optimize the parameters of the deep Q-network and reward model. Pytorch \cite{paszke2017automatic} is an open source machine learning library, built on top of Python, which provides tools to perform automatic differentiation. This allows us to compute gradient without differentiating by hand our loss function with respect to our model parameters. We implement the buffers for collecting agent experience, storing annotated clips in SciPy \cite{jones2001}, which is also built on top of Python. This gives finer control over data representation and sampling.

\chapter{Experimental Details}
%TODO does it make sense to separate experiment details from method? I think so.
% Here I'll give the experimental details
% Gym. incl. why I chose it
% Cartpole/the envs I use
% Hyperparameter settings (and all the other args e.g. number of labels acquired per round, number of repetitions etc.) (maybe put these in appendix)
% Links to code
We used the environments provided by OpenAI Gym to run our experiments.

\section{CartPole-v0}

\chapter{Results}
%TODO does it make sense to separate experiment details from results? I'm less certain.
% Here I'll describe the results in a shiny way
% Graphs and comments on whether I achieved the goal of applying active learning to increase the sample efficiency of reward modelling
\section{CartPole-v0}
%\includegraphics[scale=1]{./results_A1}

\chapter{Conclusions}
\section{Summary}
\section{Evaluation}
% critical assessment of the work that has been done and the process of doing it
% perhaps include subsections for approaches that were tried and did not work; and personal development
\section{Future Work}

% acknowledgements

\bibliographystyle{apalike}
\bibliography{library,additional}

\appendix
\appendixpage
\noappendicestocpagenum
\addappheadtotoc
\chapter{Some Appendix Material}

\end{document}